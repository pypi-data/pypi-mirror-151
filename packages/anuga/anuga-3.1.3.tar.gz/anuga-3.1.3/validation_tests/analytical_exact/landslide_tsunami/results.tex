\section{Tsunami Runup Analytical Solution}

This test case is `Benchmark Problem 1' from the `Third International workshop
on long-wave runup models, June 17-18 2004'. It models the runup of an initial
waveform on a linearly sloping beach. The analytical solution was produced 
using the techniques of Carrier, Wu and Yeh~\cite{CGY2003}. The problem descriptions 
and solution were sourced from \url{http://isec.nacse.org/workshop/2004\_cornell/bmark1.html}

\subsection{Results}

Figures~\ref{fig:stage}-\ref{fig:vel} compare the modelled and analytical stage
and velocity at various instances in time, while Figure~\ref{fig:shore} shows
the time-evolution of the shoreline position and velocity at the shoreline. For
the numerical model, the shoreline position and velocity are defined based on
two depth thresholds (see figure titles), as \anuga{} does not explicitly track
the wet-dry interface. We expect to see good agreement between the analytical
and numerical solutions for both stage and velocity over most of the domain.  A
discrepency near the shoreline can occur in some \anuga{} algorithms,
particularly in the velocity.


\begin{figure}
\begin{center}
\includegraphics[width=0.7\textwidth]{figure_stage_0.png}
\includegraphics[width=0.7\textwidth]{figure_stage_1.png}
\includegraphics[width=0.7\textwidth]{figure_stage_2.png}
\caption{Water surface elevation at several instants in time}
\label{fig:stage}
\end{center}
\end{figure}

\begin{figure}
\begin{center}
\includegraphics[width=0.7\textwidth]{figure_vel_0.png}
\includegraphics[width=0.7\textwidth]{figure_vel_1.png}
\includegraphics[width=0.7\textwidth]{figure_vel_2.png}
\caption{Velocity at several instants in time}
\label{fig:vel}
\end{center}
\end{figure}

\begin{figure}
\begin{center}
\includegraphics[width=0.7\textwidth]{Shoreline_position.png}
\includegraphics[width=0.7\textwidth]{Shoreline_velocity.png}
\caption{Timeseries of shoreline position and velocity}
\label{fig:shore}
\end{center}
\end{figure}

\subsection{Comment on algorithm-specific performance}
At the time of writing (17/05/2013), the `tsunami' algorithm has a small
phase-lag during the runup stage. This is attributed to the extra mass storage
in each numerical cell associated with that algorithm, which slightly reduces
the rate of inundation. It can be further reduced with mesh refinement.
Regardless, the solution quality is quite comparable with results using
other solvers on similar sized meshes~\cite{CCM2013,P2012}. 

The discontinuous elevation algorithms show different artefacts, related to
their tendency to have slow drainage in nearly-dry areas (so a very shallow
layer of water keeps flowing down-slope even when analytically, the slope
should have dried). This will be manifest most obviously in in velocity
aftefacts over the nearly-dry slope following draw-down. The amount of water
retained on the slope is small, so this is issue is not obvious in the stage
plots. However, it does complicate the definition of the wet-dry interface,
leading to potentially large discrepencies in the computed velocity at the
shoreline and the shoreline location.

\endinput
