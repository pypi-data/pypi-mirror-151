%Note for this test case:
%Simple water flow example using ANUGA: Water flowing down a channel.
%It was called "steep_slope" in an old validation test.


\section{Shallow flow down a mild slope with coarsely resolved topography}
This case simulates very shallow flow running down a mild slope, with coarsely resolved topography, and a very low discharge. It represents a numerically more challenging idealisation of the rainfall-runoff problem, as compared to the case with higher discharge and resolution. This case has an analytical solution, and in particular, we consider the steady-uniform solution with the values of bed slope and friction slope are the same.   

Suppose that we are given a one dimensional domain. The steady state conditions with a contant water depth everywhere make the shallow water equations to the single identity
\begin{equation}
z_x = - S_f.
\end{equation}
Here $q=uh$ is the momentum or water discharge and $S_f$ is the symbol for the force of bottom friction involving Manning's coefficient $n$. We take 
\begin{equation}
S_f = n^2 \frac{q|q|}{h^{10/3}}.
\end{equation}
If $q$, $n$, and $z_x$ are given, then the analytical solution is
\begin{equation}
u(x)= \left[- n^{-2} q^{4/3} z_x\right]^{3/10},
\end{equation}
\begin{equation}
h(x)= \frac{q}{u}\,.
\end{equation}

\subsection{Results}
For our test, we consider a square dimensional domain with length and width 100. We take $q=1/1000.$, $n=0.03$, and $z_x=-0.1$.
The topography is
\begin{equation}
z(x, y)= -0.1 x\,.
\end{equation}
The initial condition is $u=v=0$ and
\begin{equation}
w(x,y,0)= -0.1 x + 0.01\,.
\end{equation}


Some simulation results are as follows.
Figures~\ref{fig:depthdownchan} shows the steady state depth in the downstream direction. There should be a good agreement with the analytical solution in the middle of the domain (the boundaries will not reach steady-uniform flow). 
Figures~\ref{fig:xvelscrosschan} and~\ref{fig:yvelscroschan} show the steady state $x$- and $y$-velocities, along a slice in the cross slope direction (near $x=50$). In the middle of the domain the $x$-velocities should agree well with the analytical solution, and the $y$-velocities should be zero. 

\begin{figure}
\begin{center}
\includegraphics[width=0.8\textwidth]{depth_x.png}
\caption{Depth in the downstream direction}
\label{fig:depthdownchan}
\end{center}
\end{figure}

\begin{figure}
\begin{center}
\includegraphics[width=0.8\textwidth]{xvelocity_x.png}
\caption{X velocity in the downstream direction}
\label{fig:xveldownchan}
\end{center}
\end{figure}

\begin{figure}
\begin{center}
\includegraphics[width=0.8\textwidth]{depth_y.png}
\caption{Depth in the downstream direction}
\label{fig:depthacrosschan}
\end{center}
\end{figure}
 

\begin{figure}
\begin{center}
\includegraphics[width=0.8\textwidth]{x_velocity.png}
\caption{$x$-velocity along the cross-section $x=50$ (i.e. a cross-section with constant bed elevation)}
\label{fig:xvelscrosschan}
\end{center}
\end{figure}

\begin{figure}
\begin{center}
\includegraphics[width=0.8\textwidth]{y_velocity.png}
\caption{$y$-velocity along the cross-section $x=50$ (i.e. a cross-section with constant bed elevation)}
\label{fig:yvelscroschan}
\end{center}
\end{figure}


\endinput
